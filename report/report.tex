\documentclass{article}
\usepackage[utf8]{inputenc}
\usepackage[spanish, es-tabla]{babel}
\usepackage[a4paper]{geometry}
\usepackage{graphicx}
\usepackage{booktabs} 
\usepackage[table]{xcolor}
\usepackage{amsmath}
\usepackage{float}
\usepackage{hyperref}
\geometry{top=2.0cm, bottom=3.0cm, left=2.5cm, right=2.5cm}
\title{Algoritmos Bioinspirados: Evolución Diferencial}
\author{Alberto García y Diego Martínez}
\begin{document}
\maketitle
\section{Introducción}
En este trabajo hemos implementado el algoritmo diferencial desarrollado en el paper de Tian, Gao y Dai\cite{mainPaper}. Una de las principales características de este algoritmo es la capacidad de autogestión y adaptabilidad a la hora de elegir entre diversidad y convergencia.

A continuación describimos los mecanismos de mutación y recombinación, pasando por alto detalles más concretas del algoritmo como el cálculo de $F_1$ o $\vec{d}_{r2, G}$.
\subsection{Mutación}
Los individuos mutados se generan mediante la siguiente fórmula:
\begin{equation}
    \vec{v}_{i,G} = \left\{\begin{array}{ll}
            \vec{x}_{\text{rand},G} + F_1(\vec{x}_{g,G}-\vec{x}_{\text{rand}, G})+ F_2(\vec{x}_{r1,G}-\vec{d}_{r2, G})&\text{si rand}<\xi_1\\
            \vec{x}_{\text{cur},G} + F_1(\vec{x}_{g,G}-\vec{x}_{\text{cur}, G})+ F_2(\vec{x}_{r1,G}-\vec{d}_{r2, G})&\text{caso contrario}
    \end{array}\right.
\label{ec_mutacion}
\end{equation}
Donde las variables tienen el siguiente significado:
\begin{itemize}
    \item $\xi_1$: Constante definida por el usuario, $\le1$.
    \item rand: Variable aleatoria con probabilidad uniforme entre 0 y 1.
    \item $G$: Generación a la que pertenecen los individuos.
    \item $\vec{v}_{i,G}$: Individuo mutado.
    \item $\vec{x}_{\text{cur},G}$: Individuo a mutar.
    \item $F_1,\ F_2$: Constantes determinadas por el fitness.
    \item $\vec{x}_{\text{rand},G}$: Individuo seleccionado al azar.
    \item $\vec{x}_{g,G}$: \textit{Guiding individual}. Individuo seleccionado al azar entre los individuos con mejor \textit{fitness}.
    \item $\vec{d}_{r2,G}$ es un vector aleatorio dentro del espacio de búsqueda.
\end{itemize}
La predisposición hacia la convergencia o la diversidad viene dada por el valor que demos a $\xi_1$.

El primer término favorece la convergencia: ``sustituye'' la posición del individuo original por uno aleatorio, forzando que los puntos se mantengan donde está la mayoría.

El segundo término favorece la exploración: ``mantiene'' la posición original y luego le suma dos perturbaciones: una que lo lleva hacia el \textit{guiding individual} y otra aleatoria.

Nótese que no se están redifiniendo las posiciones de la población original, denotada por $\vec{x}$, sino definiendo una nueva población mutada $\vec{v}$.
\section{Estructura del programa}
Para escribir el programa hemos modificado bastante el programa inicial dado en clase. El programa se sigue llamando desde \texttt{lanzador.R}, que inicializa el problema mediante las funciones del directorio \texttt{funciones} y el script \texttt{inicia.R} (aunque este ha sido renombrado como \texttt{inicializador.R}).

A partir de aquí los scripts se han directamente sustituido o eliminado. Una vez inicializado el problema, \texttt{lanzador.R} llama a \texttt{evolutivo.R}. Este es el script central del programa, y se encargará de llevar a cabo todos los pasos descritos en el paper de Tian, Gao y Dai\cite{mainPaper}. Para ello hará uso de los siguientes scripts:
\begin{itemize}
    \item \texttt{mutacion.R}: este script recibe como argumento una población de individuos y genera una población mutada, sin modificar la población original.
    \item \texttt{crossover.R}: este script toma la población original y la mutada mediante \texttt{mutacion.R} y las combina generando un individuo \textit{trial}, en base a la variable \textit{CR}, que determina la probabilidad de un individuo de mutar. El \textit{trial} generado es un individuo que puede tener componentes tanto del individuo original como del individuo mutado.
    \item \texttt{seleccion.R}: este script selecciona los individuos eligiendo entre los individuos originales o los \textit{trials}. Para ello tiene en cuenta el \textit{fitness} de los individuos.
\end{itemize}
\bibliography{bibliografia}{}
\bibliographystyle{ieeetr}
\end{document}
